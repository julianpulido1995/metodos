\documentclass{article}
\usepackage{graphicx}

\begin{document}
\title{Resultados Tarea 4 Metodos Computacionales}
\author{Julian Pulido}
\date{ 29 de Abril de 2017}
\maketitle

\section{Introduccion breve}
En este documento se presentan los resultados de la tarea 4 de metodos computacionales, presentando entonces inicialmente para el caso 1, las graficas referentes al perfil de temperatura para tiempos de 0s, 100s y 2500s para las 3 condiciones de frontera. Para el tiempo 0, la grafica sera igual en todas 3 condiciones, dado que solo se grafica la caja de lado de 1m (no se tuvo en cuenta la frontera para graficar, todo lo que hay en la caja inicialmente es lo mismo. Posteriormente se muestran las graficas del caso 2 y finalmente la temperatura media en funcion del tiempo.


\section{Caso1}
\begin{figure}
\centering
\includegraphics[scale = 0.5]{fijastiempo0caso1.png}
\caption{ Tiempo 0, Condiciones fijas caso 1}
 \end{figure}

\begin{figure}
 \centering
\includegraphics[scale = 0.5]{fijastiempo100caso1.png}
 \caption{ Tiempo 100, Condiciones fijas caso 1}
 \end{figure}

\begin{figure}
 \centering
\includegraphics[scale = 0.5]{fijastiempo2500caso1.png}
 \caption{ Tiempo 2500, Condiciones fijas caso 1}
 \end{figure}



\begin{figure}
 \centering
\includegraphics[scale = 0.5]{abiertastiempo0caso1.png}
 \caption{ Tiempo 0, Condiciones abiertas caso 1}
 \end{figure}

\begin{figure}
 \centering
\includegraphics[scale = 0.5]{abiertastiempo100caso1.png}
 \caption{ Tiempo 100, Condiciones abiertas caso 1}
 \end{figure}

\begin{figure}
 \centering
\includegraphics[scale = 0.5]{abiertastiempo2500caso1.png}
 \caption{ Tiempo 2500, Condiciones abiertas caso 1}
 \end{figure}


\begin{figure}
 \centering
\includegraphics[scale = 0.5]{periodicastiempo0caso1.png}
 \caption{ Tiempo 0, Condiciones periodicas caso 1}
 \end{figure}

\begin{figure}
 \centering
\includegraphics[scale = 0.5]{periodicastiempo100caso1.png}
 \caption{ Tiempo 100, Condiciones periodicas caso 1}
 \end{figure}

\begin{figure}
 \centering
\includegraphics[scale = 0.5]{periodicastiempo2500caso1.png}
 \caption{ Tiempo 2500, Condiciones periodicas caso 1}
 \end{figure}

\section{Caso 2}
\begin{figure}
 \centering
\includegraphics[scale = 0.5]{fijastiempo0caso2.png}
 \caption{ Tiempo 0, Condiciones fijas caso 2}
 \end{figure}

\begin{figure}
 \centering
\includegraphics[scale = 0.5]{fijastiempo100caso2.png}
 \caption{ Tiempo 100, Condiciones fijas caso 2}
 \end{figure}

\begin{figure}
 \centering
\includegraphics[scale = 0.5]{fijastiempo2500caso2.png}
 \caption{ Tiempo 2500, Condiciones fijas caso 2}
 \end{figure}



\begin{figure}
 \centering
\includegraphics[scale = 0.5]{abiertastiempo0caso2.png}
 \caption{ Tiempo 0, Condiciones abiertas caso 2}
 \end{figure}

\begin{figure}
 \centering
\includegraphics[scale = 0.5]{abiertastiempo100caso2.png}
 \caption{ Tiempo 100, Condiciones abiertas caso 2}
 \end{figure}

\begin{figure}
 \centering
\includegraphics[scale = 0.5]{abiertastiempo2500caso2.png}
 \caption{ Tiempo 2500, Condiciones abiertas caso 2}
 \end{figure}


\begin{figure}
 \centering
\includegraphics[scale = 0.5]{periodicastiempo0caso2.png}
 \caption{ Tiempo 0, Condiciones periodicas caso 2}
 \end{figure}

\begin{figure}
 \centering
\includegraphics[scale = 0.7]{periodicastiempo100caso2.png}
 \caption{ Tiempo 100, Condiciones periodicas caso 2}
 \end{figure}

\begin{figure}
 \centering
\includegraphics[scale = 0.5]{periodicastiempo2500caso2.png}
 \caption{ Tiempo 2500, Condiciones periodicas caso 2}
 \end{figure}

\begin{figure}
 \centering
\includegraphics[scale = 0.5]{tmediacaso1.png}
 \caption{ Temperatura media caso 1}
 \end{figure}
\begin{figure}
 \centering
\includegraphics[scale = 0.5]{tmediacaso2.png}
 \caption{ Temperatura media caso 2}
 \end{figure}



\end{document}
